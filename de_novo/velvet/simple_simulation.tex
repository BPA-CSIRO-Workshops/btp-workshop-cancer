\subsection{Simple Assembly Simulation}

\begin{note}
The data for this section is from Staphylococcus aureus MRSA252, a genome
closely related to the genome that provided the short read data in the earlier
sections of this exercise. The sequence data this time is the fully assembled
genome. The genome size is therefore known exactly and is 2,902,619 bp.
\end{note}

\begin{information}
In this exercise you will process the single whole genome sequence with \texttt{velveth}
and \texttt{velvetg}, look at the output only and go no further. The main intent of
processing this whole genome is to compute its N50 value. This must clearly be
very close to the ideal N50 value for the short reads assembly and so aid
evaluation of that assembly.
\end{information}

\begin{steps}
To begin, move back to the main directory for this exercise, make a
sub-directory for the processing of this data and move into it. All in one go,
this would be:
\begin{lstlisting}
cd ~/NGS/velvet/part1/ 
mkdir MRSA252 
cd MRSA252
\end{lstlisting}

Next, you need to download the genome sequence from
\url{http://www.ensemblgenomes.org/} which holds five new sites, for bacteria,
protists, fungi, plants and invertebrate metazoa. You could browse for the data
you require or use the file which we have downloaded for you. For the easier of
these options, make and check a symlink to the local file and with the
commands:
\begin{lstlisting}
ln -s ~/NGS/Data/s_aureus_mrsa252.EB1_s_aureus_mrsa252.dna.chromosome.Chromosome.fa.gz ./
ls -l
\end{lstlisting}

\end{steps}

\begin{note}
Usually Velvet expects relatively short sequence entries and for this reason has
a read limit of 32,767 bp per sequence entry. As the genome size is 2,902,619 bp
- longer as the allowed limit and does not fit with the standard settings into
velvet. But like the maximum k-mer size option, you can tell Velvet during
compile time, using \texttt{LONGSEQUENCES=Y}, to expect longer input sequences
than usual. I already prepared the executable which you can use by typing
\texttt{velveth\_long} and \texttt{velvetg\_long}.
\end{note}

\begin{steps}
Now, run \texttt{velveth\_long}, using the file you either just downloaded or created a
symlink to as the input:
\begin{lstlisting}
velveth_long run_25 25 -fasta.gz -long s_aureus_mrsa252.EB1_s_aureus_mrsa252.dna.chromosome.Chromosome.fa.gz
velvetg_long run_25
\end{lstlisting}

\end{steps}

\begin{questions}
What is the N50?
\begin{answer}
24,142 bp
\end{answer}

How does the N50 compare to the previous single end run (SRS004748)?
\begin{answer}
Big difference
\end{answer}
  
Does the total length differ from the input sequence length?
\begin{answer}
2,817,181 (stats) vs 2,902,619 (input)
\end{answer}
  
What happens when you rerun velvet with a different k-mer length?
\begin{answer}
K-mer 31: N50: 30,669 bp, total 2,822,878
\end{answer}

\end{questions}